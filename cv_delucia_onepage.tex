%\documentclass[11pt,twoside,a4paper]{letter}
\documentclass[fancyheadings,12pt,a4paper]{article}
%\documentclass[12pt,a4paper]{book}
%\usepackage{bookman}
\usepackage{amsmath}
\usepackage{job}
\include{CVmacros}
\begin{document}


\begin{large}
\noindent {\bf Gabriella De Lucia}
\end{large}

\smallskip\smallskip\noindent {\bf \underline{Current position}:}
Senior researcher at INAF Astronomical Observatory of Trieste (OATs).

\smallskip\smallskip\noindent {\bf  \underline{Role in the project}:}
Development of theretical models of galaxy formation in a cosmological context
to be used to interpret observational data. 

\smallskip\smallskip\noindent {\bf \underline{Education}:} {\it Laurea
  in Fisica} at the University ``Federico II'' of Naples in 2000; {\it Doctor
  Rerum Naturalium} (Ph.D.), at the Ludwig-Maximilian Universit\"at of
M\"unchen in 2004.

\smallskip\smallskip\noindent {\bf  \underline{Positions held}:} {\it
  Research contract} at the Astronomical Observatory of Capodimonte (5 months
in 2000); {\it Graduate student} at the Max-Planck Institute for Astrophysics
in Garching bei M\"unchen from 2001 to 2004; {\it Long term PostDoctoral
  Fellowship} at MPA from 2004 to 2009; Primo Ricercatore (fixed term position
funded by ERC) at INAF-OATs from February 2009 to January 2014; Researcher
(permanent) at INAF-OATs since 2013.

\smallskip\smallskip\noindent {\bf  \underline{Honors}:} MERAC Prize
for the Best Early Career Researcher in Theoretical Astrophysics (awarded by
the European Astronomical Society) in 2013; Order of Merit (Officer) of the
Italian Republic (awarded by the President of the Italian Republic) in 2011;
ERC Starting Independent Researcher Grant (success rate of the call $\sim 3$\%)
in 2008.

\smallskip\smallskip\noindent {\bf \underline{Professional Activities
      (last 5 years)}:} Frequent Peer Reviewer for {\it Monthly Notices of the
  Royal Astronomical Society} (main journal and letters), {\it The
  Astrophysical Journals} (main journal and letters), {\it Astronomy \&
  Astrophysics}; Time Allocation Committee for the telescopes TNG/LBT/REM
(2014-2016); ESO Observing Programmes Committee (Periods 91 and 92); Expert
Reviewer for Physics Discovery Grants for the Natural Science Research Council
of Canada (2016), ERC Starting 2015 call, ERC Advanced 2013 call, and the
French Research Agency (2012). 

\smallskip\smallskip\noindent {\bf  \underline{Conferences and
      Seminars}:} 16 invited reviews at International Conferences; 14 invited
talks at International Conferences; 25 invited seminars and colloquia. 

\smallskip\smallskip\noindent {\bf \underline{Publications}:} 135
refereed papers in peer-reviewed journals with more than 9700 citations
(excluding self-citations). 18 refereed publications as first author and 27 as
second author. Selected publications:
\begin{itemize}
\item
\emph{Galaxy assembly, stellar feedback and metal enrichment: the view from the
  GAEA model}, M. Hirschmann, {\bf G. De Lucia}, F. Fontanot, 2016, MNRAS, 461,
1760\vspace{-0.18truecm}
\item
\emph{Elemental abundances in Milky Way like galaxies from a hierarchical
  galaxy formation model}, {\bf G. De Lucia}, L. Tornatore, C.S. Frenk,
A. Helmi, J.F. Navarro, S.D.M. White, 2014, MNRAS, 444, 970\vspace{-0.18truecm}
\item
\emph{The environmental history of group and cluster galaxies in a $\Lambda$CDM
  Universe}, {\bf G. De Lucia}, S. Weinmann, B. Poggianti, A. Aragon-Salamanca,
D. Zaritsky, 2012, MNRAS, 423, 1277\vspace{-0.18truecm}
\item
\emph{The hierarchical formation of the brightest cluster galaxies}, {\bf
  Gabriella De Lucia} and J\'er\'emy Blaizot, 2007, MNRAS, 375,
2\vspace{-0.18truecm}
\item
\emph{The Buildup of the Red Sequence in Galaxy Clusters since z $\sim 0.8$},
     {\bf G. De Lucia}, B. M. Poggianti, A. Arag{\'o}n-Salamanca, D. Clowe,
     C. Halliday, P. Jablonka, B. Milvang-Jensen, R. Pell{\'o}, S. Poirier,
     G. Rudnick, R. Saglia, L. Simard, S. D. M. White, 2004, ApJ Letters, 610,
     77

\end{itemize}
\end{document}

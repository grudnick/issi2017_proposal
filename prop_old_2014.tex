%\documentclass[11pt,twocolumn]{article}
%\documentclass[11pt,letter]{article}
\documentclass[11pt]{article}
%\documentclass[10pt]{aastex}

%\documentclass[preprint]{aastex}
\usepackage{graphicx,epsfig,natbib,multicol}
%\usepackage{natbib,epsfig}
%\usepackage{/Users/grudnick/LaTeX/sttools/stfloats}
%\usepackage{fullpage}
%\usepackage{multicol}
\usepackage{wrapfig}
\usepackage{times}
%\usepackage[top=0.5in, bottom=1.5in, left=1.1in, right=0.8in]{geometry}
%% \addtolength{\oddsidemargin}{-.875in}
%% 	\addtolength{\evensidemargin}{-.875in}
%% 	\addtolength{\textwidth}{1.75in}

%% 	\addtolength{\topmargin}{-.875in}
%% 	\addtolength{\textheight}{1.75in}

\usepackage{simplemargins}
\setleftmargin{1.05in}
	\setrightmargin{1.05in}
	\settopmargin{1in}
	\setbottommargin{1.2in}
%	\setallmargins{dimen}
%\setallmargins{1.2in}

%\usepackage[top=0.5in, bottom=1.6in, left=1.1in, right=0.8in]{geometry}
%\usepackage[top=0.7in, bottom=1.2in, left=1.1in, right=1.1in]{geometry}
%\usepackage{onecolfloat}
%\usepackage{epsf}
%\includeonly{references}
\usepackage{/Users/grudnick/LaTeX/macros_rudnick}
\newcommand{\HRule}{\rule{\linewidth}{0.3mm}}
\setlength{\parskip}{3pt}
\setlength{\parsep}{0pt}
\setlength{\headsep}{0pt}
\setlength{\topskip}{0pt}
\setlength{\topmargin}{0pt}
\setlength{\topsep}{0pt}
\setlength{\partopsep}{0pt}


%\addtolength{\parskip}{-4pt}

%\usepackage{pslatex}

\usepackage[compact]{titlesec}
%\titlespacing{\section}{0pt}{*0.2}{*0.2}
%\titlespacing{\subsection}{0pt}{*0.1}{*0.1}
%\titlespacing{\subsubsection}{0pt}{*0.1}{*0.1}

%% %to get EPSCoR page numbering 
%% \usepackage{fancyhdr}
%% \pagestyle{fancy}
%% \fancyhead{}
%% \fancyfoot{}
%% \cfoot[]{C-\thepage}
%% \renewcommand{\headrulewidth}{0pt}

%% % Alter some LaTeX defaults for better treatment of figures:
%%     % See p.105 of "TeX Unbound" for suggested values.
%%     % See pp. 199-200 of Lamport's "LaTeX" book for details.
%%     %   General parameters, for ALL pages:
%%     \renewcommand{\topfraction}{1.0}	% max fraction of floats at top
%%     \renewcommand{\bottomfraction}{1.0}	% max fraction of floats at bottom
%%     %   Parameters for TEXT pages (not float pages):
%%     \setcounter{topnumber}{2}
%%     \setcounter{bottomnumber}{2}
%%     \setcounter{totalnumber}{4}     % 2 may work better
%%     \setcounter{dbltopnumber}{2}    % for 2-column pages
%%     \renewcommand{\dbltopfraction}{0.9}	% fit big float above 2-col. text
%%     \renewcommand{\textfraction}{0.07}	% allow minimal text w. figs
%%     %   Parameters for FLOAT pages (not text pages):
%%     \renewcommand{\floatpagefraction}{0.7}	% require fuller float pages
%% 	% N.B.: floatpagefraction MUST be less than topfraction !!
%%     \renewcommand{\dblfloatpagefraction}{0.7}	% require fuller float pages

%% 	% remember to use [htp] or [htpb] for placement

% Figures within a column...
\makeatletter
\newenvironment{tablehere}
{\def\@captype{table}}
{}
\newenvironment{figurehere}
{\def\@captype{figure}}
{}
\makeatother

\bibpunct[;]{(}{)}{;}{a}{}{,}


\title{A Rise Before the Fall: Are the Star Formation Rates of Galaxies Boosted When They Enter A Cluster Environment?}  \author{Gregory Rudnick}
\begin{document}
%\maketitle
%\noindent\HRule

%\renewcommand{\thepage}{D--\arabic{page}}



%\hline
%\begin{center}
%Project description
%\end{center}
%\hline

%\begin{multicols}{2}

%\HRule
%\vspace{-1.5in}
\begin{center}
\large{Proposal for an ISSI International Team}\\
\Large{The Effect of Dense Environments on Gas in Galaxies over 10 Billion Years of Cosmic Time}\\
\medskip
\vspace{-0.2cm}
\large{Coordinator: Gregory Rudnick}\\
\end{center}
%\HRule

One of the main quests of modern astrophysical research is to
understand how galaxies evolve throughout cosmic time.  Since the
1930's, astronomers began to suspect that the places in which galaxies
live played an important role in their evolution.  It was observed
that galaxies residing within groups and clusters of galaxies are
systematically different than those residing in more sparsely
populated regions of the universe. In particular, they have a higher
fraction of old stars, and tend not to produce many new stars.
However, despite tremendous effort, it is still not clear by what
mechanisms galaxies are altered by their surroundings.  What has
become clear is that solving this problem requires a direct measure of
the gas, which is the fuel of star formation.  We need to know how
much gas is in galaxies, its physical state, its spatial distribution,
and how these change both over time and in different environments.  By
understanding how the fuel supply is regulated we will therefore be
able to understand the observed changes in the star formation rates.
Recent advances in instrumentation are now giving us, for the first
time, the ability to study the physical conditions of the gas over the
relevant scales in environment and cosmic time.

We propose to form a new research team to capitalize on these emergent
observational resources, synthesize the results from existing studies,
form a comprehensive picture of how gas is altered in dense
environments, and plan for the exploitation of future facilities.  Our
team is composed of 12 individuals from 6 countries whose research
expertise are as follows:
\vspace{-0.15in}
\begin{itemize}
\item \textit{Molecular tracers of star-forming gas:} F. Walter (D), F. Combes (F),
  P. Jablonka (CH), G. Rudnick (USA), T. Rawle (ES)
\vspace{-0.1in}
\item \textit{Ionized tracers of star-forming gas:} A. Arag\'{o}n-Salamanca (UK), B. Weiner (USA), B. Poggianti (IT), R. Finn (USA)
\vspace{-0.1in}
\item \textit{Dust-obscured star formation:} R. Finn (USA), V. Desai (USA), G. Rudnick (USA), T. Rawle (ES)
\vspace{-0.1in}
\item \textit{Galaxy environment:} D. Zaritsky (USA), B. Poggianti (IT), V. Desai (USA), P. Jablonka (CH), A. Arag\'{o}n-Salamanca (UK)
\vspace{-0.1in}
\item \textit{Very distant galaxy clusters:} G. Rudnick (USA), B. Weiner (USA), F. Walter (D)
\vspace{-0.1in}
\item \textit{Theoretical modeling:} G. De Lucia (IT)
\end{itemize} 
\vspace{-0.1in}
These individuals are using the largest ground-based telescopes, the
best space facilities, and the largest and newest radio
interferometers on the planet.  Together they account for much
of the data on gas in dense environments, and they are studying galaxies
over extremely large baselines in cosmic time.  

This ISSI project will bring together the work of these individuals to
form a coherent picture of how gas is affected by the environment.
Our immediate goals are to:
\vspace{-0.15in}
\begin{itemize}
\item Determine the response of the ionized gas to the environments by
  combining results of HST spectroscopic observations of intermediate redshift
  galaxy clusters with Gran-TeCan 10-meter Telescope (GTC) tunable filter
  observations of local galaxy clusters.
\vspace{-0.1in}
\item Determine how the star formation rate and dust content are
  affected by environment through the use of \spitzer\ observations of
  dust emission and millimeter-wave interferometric observations of
  the molecular gas and dust continuum in galaxy clusters spanning 10~Gyr of
  cosmic time.
\vspace{-0.1in}
\item Compare the state of the gas to that of the stars to constrain
  the transformative mechanism.
\vspace{-0.1in}
\item Initiate new observational programs to measure the physical
  state (temperature and density) and spatial distribution of the
  molecular gas and compare it to the state of the ionized gas.
\vspace{-0.1in}
\item Make powerful new constraints on the implementation of gas
  depletion in galaxy formation models.
\end{itemize}
\vspace{-0.1in}
%This program will produce breakthroughs in our understanding by
%bringing together individuals who each are leading various elements of
%the above analysis.

\centerline{{\bf \underline{ The Role of Environment in Galaxy Evolution}}}
\medskip

\indent{\bf Galaxy Clusters as Executioners?} Galaxies in dense environments
have lower average star-formation rates (SFRs) than field galaxies out
to at least $z \sim 1$ \citep[e.g.][]{Poggianti99,Lewis02,Gomez03,Postman05}.
However, despite tremendous observational effort, it
is still not clear whether the clusters actively alter the gas content
of infalling galaxies or whether they are the final resting place of
dead galaxies whose gas was depleted before entering the cluster
environment.

Studies in the local Universe have found
that the average SFR starts to decline at group densities, which are
comparable to the density at $3-4$ times the cluster virial radius
\citep{Lewis02,Gomez03}.  This implies that processing prior to
accretion into the cluster is important.  In contrast, spiral galaxies
in the Virgo cluster show evidence of cold gas stripping and truncated
star forming disks
\citep{Koopmann98,Koopmann04,Dale01,Crowl05,Chung07}.  This
demonstrates that the cluster environment is actively altering the
star-formation properties of infalling galaxies.

One reason for the continuing ambiguity is that most groups have
focused on studying only one phase of the gas in galaxies
(e.g. ionized vs. cold).  To conclusively determine the cause for the
end of star formation in dense environments it is important to study
the fuel of star formation itself.  In addition, because galaxies may
be altered well outside of the cluster core, it is crucial to probe
intermediate (i.e. group) environments through which galaxies pass on
their way into the cluster.

\textbf{The importance of understanding the gas:} Through decades of
work it has become clear that local supply of cold (mostly molecular)
gas determines the SFR of galaxies
\citep{Kennicutt98b,Bigiel08,Leroy08}. Therefore, to truly understand
how the SFRs of galaxies are altered it is necessary to directly probe
the content and spatial distribution of the gas.

Different phases of the gas yield complementary information.  The
content and distribution of ionized gas traces the star formation that
is relatively unobscured by dust while the mid-infrared emission tells
us about the obscured star formation.  Likewise, observations of the
molecular gas constrain the fuel of star formation and its physical
conditions, i.e. temperature, density, and filling factor.  Although
the mass of molecular gas is dominated by $H_2$, which has little
accessible emission due to its lack of a permanent dipole moment, we
can use well-calibrated proxies, such as rotationally excited CO.

It is important to study the amount of gas and its spatial
distribution with respect to that of the stars.  A key difference
between the Virgo and other local results mentioned above is that the
Virgo studies are based on spatially resolved \ha\ or HI maps while
the others are insensitive to processes that preferentially affect the
outer radii of galaxies.  This is an important distinction as the most
common mechanisms proposed for depleting gas make different
predictions for the size and symmetry of the gas and stars disks.

In addition, studying spatially-resolved gas and stellar disks {\em in
  both groups and clusters} can help constrain the relative importance
of these physical processes because their effectiveness varies with
the density of the intra-cluster or intra-group medium and the
velocity of galaxies relative to each other and the intra-cluster
medium (ICM).  For example, the removal of cold disk gas through
ram-pressure stripping is expected to be most effective in the cluster
core, whereas galaxy-galaxy interactions, which exhaust the gas supply
through a burst of star-formation, are most effective in groups or
cluster outskirts.

A preliminary study has found that that star forming galaxies in
clusters appear to have had some of their molecular gas stripped while
leaving their SFRs - fueled by the densest gas - unaffected for a time
\citep{Jablonka13}.  While suggestive, these conclusions are based
primarily on local samples, which have numerous problems.
Intermediate redshift is the place to look for environmental effects
because galaxies are still actively evolving.  However, the number of
CO measurements at these redshifts is very small and our efforts at
enlarging them have just started.  

%are only based on a handful
%of CO detections at intermediate redshift and do not contain
%information about the spatial distribution of gas in the galaxies nor
%about the dependence of gas properties on location in the cluster.%

\textbf{A revolution in our knowledge of the gas:} We are entering a
new age of possibility in our ability to study how gas is affected by
dense environments thanks to space-based (\spitzer, \herschel, HST,
JWST) and ground-based (ALMA, IRAM-interferometer, JVLA, GTC, VLT)
observatories.  With tunable filter and slitless spectroscopy
observations with the GTC and HST we are obtaining detailed spatial
maps of star formation in clusters and their infall regions for
systems at $z<0.6$.  HST grism spectroscopy is allowing us to probe
the ionized gas and stellar populations of a sample of the most
distant clusters at redshifts $z>1.5$.  \spitzer\ and \herschel\ have
also given us views of the obscured star formation of galaxies back to
the early epochs of time.  Locally it allows us to map the relative
spatial distribution of the dust and the stars, thus tracing the
stripping of the cold gas.  Millimeter observations at the JVLA, IRAM,
and ALMA are enabling us to directly measure the content of molecular
gas in cluster galaxies.  Finally, the fully commissioned ALMA promises us
the ability to measure the spatially resolved gas excitation in
galaxies, thus directly probing the physical conditions of star
formation as they interact with dense environments.

Each of our proposed ISSI team members is leading 1--2 of the efforts
described above.  We will bring together these resources to
adress the following questions: In what environment moving from the
field, through groups, and into cluster cores is the gas of galaxies
first affected?  How are the content, distribution, density,
and temperature of the gas altered?  Over what timescales does the
depletion occur?  What are the responsible mechanisms for the gas
removal and do they operate differently in different environments?

\centerline{{\bf \underline{ Our proposed activities}}}
\medskip

The first step will be to study specific phases of the gas.  The
analysis of individual datasets, which usually focus on an indidvual
environment, redshift, or gas phase, will be primarily done by the PIs
of those programs.  Collectively our team members are studying the
densest environments over 10~Gyr of cosmic time.  However, until we
try to bring all of the results together it will be impossible to form
coherent and broad-reaching conclusions with the data that have been
provided by the revolutionary new facilities.  Performing this
synthesis is one of the main goals of our proposed ISSI team.

\textit{Ionized gas:} The effect of environment on the dense ionized
gas will be determined by combining an HST/WFC3 grism Cycle 20 program
(PI: Rudnick) on the infall regions of 4 intermediate redshift
clusters with a narrow-band imaging ESO Large Programe with the GTC
(PI: Arag\'on-Salamanca) on a z=0.2 supercluster.  With these programs
we will distinguish between different methods of gas depletion by
determining relative structure of the star-forming and stellar disks.
For example, ram-pressure stripping makes the prediction that the gas
disks will be asymmetric \citep[e.g.][]{Quilis00,Crowl05} with respect
to the stars and that the SFRs in the inner parts of the galaxy should
be the same as or even enhanced with respect to field galaxies
\citep{Koopmann04,Weinmann10}.  Galaxy-galaxy interactions, however,
will result in both the gas and stars being asymmetric.  Finally,
starvation, which describes a weaker version of ram-pressure stripping
\citep[e.g.][]{Larson80}, may only affect the relative sizes of the
gas and stellar disks.  By comparing the \ha\ properties of the
cluster galaxies and those in infalling groups, we will be able to
determine if the cluster environment plays an important role in
altering star-forming galaxies and how how that role has evolved over
the last 5~Gyr.
%.  By combining these studies over a large range in redshift and
%cluster or group mass our ISSI team will be able to determine how
%stripping processes have evolved over the last 5~Gyr.

\textit{Obscured star formation:} Infrared observations from
\spitzer\ and \herschel\ probe the emission by dust grains that have
been heated by star formation and can by used to infer the total
infrared luminosity \lir\ and SFR.  Using measurements for clusters at
$z<2$ it is clear that the total SFR in cluster galaxies has declined
faster than the field \citep{Finn10,Saintonge08,Tran10,Alberts14}.
This could be because of ``accelerated evolution'' in clusters
\citep{Papovich12} or because the cluster environment speeds up the
depletion of gas at late times.  To break this degeneracy it is
necessary to look in a more detailed way at the properties of the gas
itself.

To address this we will use completed wide-field
\spitzer\ 24\micron\ observations around 10 clusters at $0.6<z<0.8$
(PI Rudnick) to measure whether there is a location at which the
fraction of vigorously star-forming galaxies drops.  We will also use
\spitzer\ 24\micron\ observations of 9 local groups and clusters to
compare the spatial distribution of the dust emission in galaxies to
that of their stars (PI Finn).  Active stripping of the cold gas will
result in a spatial offset of the 24\micron\ emission from the stellar
light, while a mere decoupling of the galaxy from its gas supply will
result in a lower mean intensity and smaller 24\micron\ size.  The
ISSI team will combine these two approaches to determine what the
spatial distribution of the dust is at the same density or
clustercentric radius where a suppression of star formation is seen.

\textit{The fuel for star formation.} To truly understand the
modulation of star formation by reduction in the $H_2$ fuel supply
requires that we observe the gas, or at least the best tracer
possible, which is CO.  Thanks to dramatic technological advances in
millimetric interferometry with JVLA, IRAM, and ALMA we are now able
to study CO over the full redshift range in which clusters are
growing.  As described above, we have already started on this rode by
starting to increase the number of CO detections in intermediate
redshift cluster galaxies \citep{Jablonka13}.  Such measurements are
crucial as they have less systematics then lower-redshift measurments
and are being made when the evolutionary rates and gas contents of
galaxies were much higher.  Our ISSI team will use these facilities to
build large samples of galaxies at intermediate redshift.  As the
result of a new large JVLA program, team members have now detected CO
in 5 galaxies that reside in a well characterized $z=1.62$ cluster
(Rudnick et al. in prep).  Taken together with the intermediate
redshift measurements, these programs comprise most of the CO
detections in cluster galaxies outside the nearby universe.  Part of
our ISSI program will be to combine these studies to measure how
molecular gas and star formation relate in dense environments since
$z<2$.  CO measurements for field galaxies at normal IR luminosities
are rapidly increasing, promising a perfect field comparison sample.
This is a particularly important part of our project as it gets to the
heart of what is driving star formation and its cessation, namely the
molecular gas and its depletion.

\textit{Providing crucial constraints for theoretical models.} All of
these observations will give us an unprecedented view of how gas is
being altered in galaxies as they enter cluster environments.  Our
ISSI team will use them to to place very strong constraints on
theoretical models of galaxy formation.  In nearly all of these
models, the gas supply to galaxies is cut off upon their entry to
another more massive dark matter halo.  The exact mechanism for this
``satellite quenching'' is unspecified in the models but is
parameterized as a timescale for the cutoff of the gas supply.  Our
observational programs will directly constrain these models by revealing where the gas supply is cut off, i.e. the virial radius, cluster
core, in groups, and how fast it is shut off.  Current
attempts at constraining these timescales have revolved around
modeling the buildup of quiescent galaxies and result in uncomfortably
long quenching timescales \citep{McGee11,DeLucia12a}. The fundamental
problem is that the models don't properly treat quenching because the
mechanism is unknown.  We will make crucial steps towards answering
this long-standing question.

\textit{Elucidating measurements with revolutionary facilities:}
Future facilities like JWST and those nearing completion, like ALMA,
will provide us an unprecedented opportunity to study the gas contents
of galaxies as a function of environment.  ALMA will open the door to
spatially resolved studies of the gas excitation and JWST will allow
the high spatial resolution of the ionized gas.  Our ISSI team will
build large programs on these facilities to tackle the main questions
of this proposal.  A timely effort in the design of these programs is
crucial, as JWST has a limited mission lifetime.  Likewise, as ALMA
ramps up to full sensitivity over the next few years, our ISSI team
will be in a perfect position to propose large programs to study the
cold gas.  As more distant clusters become confirmed, a priority of
the team will be observing these with HST, ALMA, and eventually JWST.

\textbf{Why is our project powerful?}  Collectively, our team has
access to the ideal data to address the questions outlined in our
proposal.  Our cluster sample spans $\sim10~$Gyr of cosmic time and
includes the highest redshift cluster with both extremely deep HST
grism data and JVLA data.  Importantly, our clusters are ideally
suited for evolutionary studies as they are all typical progenitors of
our local clusters (Milvang-Jensen et al. 2008; Rudnick et al. 2012).
At intermediate redshift we probe far enough out in clustercentric
radius to identify all of the members that will end up in the cluster
at $z=0$ (Just et al. 2014).

Our proposed collaboration will also probe nearly all of the phases of
the gas that are relevant for star formation, from the hot dense gas
that traces active star formation to the cold molecular gas that is
the fuel of star formation.  By combining our measurements with
theoretical models we will gain a greatly improved understanding of
how star formation is regulated, and eventually quenched, in dense
environments.  This synergy of the appropriate data, in the
appropriate sample, spanning a large range in redshift, and with
accompanying theoretical modeling is unique.

\textbf{Our proposed team:} Our team is composed of 12 highly
recognized experts in various areas of galaxy evolution studies and
represents 6 countries.  They are playing key roles or are leading
projects in one or more of the areas mentioned
above. $\bullet$~\textit{Molecular tracers of star-forming gas:}
F. Walter (D), F. Combes (F), P. Jablonka (CH), G. Rudnick (USA),
T. Rawle (ES) $\bullet$~\textit{Ionized tracers of star-forming gas:}
A. Arag\'{o}n-Salamanca (UK), B. Weiner (USA), B. Poggianti (IT),
R. Finn (USA) $\bullet$~\textit{Dust-obscured star formation:} R. Finn
(USA), V. Desai (USA), G. Rudnick (USA), T. Rawle (ES)
$\bullet$~\textit{Galaxy environment:} D. Zaritsky (USA), B. Poggianti
(IT), V. Desai (USA), P. Jablonka (CH), A. Arag\'{o}n-Salamanca (UK)
$\bullet$~\textit{Very distant clusters:} G. Rudnick (USA), B. Weiner
(USA), F. Walter (D) $\bullet$~\textit{Theoretical modeling:} G. De
Lucia (IT)

\textbf{The value of ISSI.}  Historically, studies of the cold gas
have been carried out independently from those that study the effects
of environment.  Likewise, the high redshift cluster community has,
been focused on finding systems, with relatively little work having
been done on the effect of environment on the physical properties of
galaxies.  Making significant progress requires a concerted and
multi-wavelength approach that stretches across large swaths of cosmic
time.  This is a highly valued endeavor as understanding the gas in
galaxies was highlighted in the Astro2010 Decadal report from the
U.S. National Academy of Sciences.  It is also a main focus of ALMA,
the largest Europe-US-Japan project of the decade. The funding from
ISSI gives us a unique opportunity to share our results in the context
of a larger picture of how gas in galaxies behaves.  Extended
face-to-face meetings are critical for such undertakings and the ISSI
funds make this possible as none of the collaborators has the
necessary funds from other sources.

\textbf{Outcomes.}  Our collaboration will result in mutiple high
impact papers describing the results above.  We will also write a
paper that combines the different studies above into a summary and
synthesis of all we know observationally about the gas in galaxies in
dense environments.  Another paper will combine those observational
constraints with our theoretical modeling to constrain the timescales
and physical mechanisms for the quenching of star formation.

We will submit multiple telescope proposals to ALMA, JVLA, PdBI, HST,
and eventually JWST to characterize the gas in much larger samples
than is currently possible.

\textbf{Schedule.} We propose to hold an initial full team (12 members)
meeting of five days to kick off the project during the summer of
2015. This would be followed by a final 5 day full team meeting in
the summer of 2016.

\textbf{Financial Support.}  We request the standard support provided
by ISSI of a per diem for the living expenses of Team members while
residing in Bern and for the travel expenses of the coordinator
(Rudnick). We would also appreciate benefiting from the ISSI Young
Scientist scheme for two young researchers.

\textbf{Required Facilities.}  We require only meeting facilities and
reasonably fast internet access.

\footnotesize{{Alberts}, S. {et~al.} 2014, \mnras, 437, 437; {Bigiel}, F. {et~al.} 2008, \aj, 136, 2846; {Chung}, A. {et~al.} 2007, \apjl, 659, L115; {Crowl}, H.~H. {et~al.} 2005, \aj, 130, 65; {Dale}, D.~A. {et~al.} 2001, \aj, 121, 1886; {De Lucia}, G. {et~al.} 2012, \mnras, 423, 1277; {Finn}, R.~A. {et~al.} 2010, \apj, 720, 87; {G{\' o}mez}, P.~L. {et~al.} 2003, \apj, 584, 210; {Jablonka}, P. {et~al.} 2013, \aap, 557, A103; {Kennicutt}, Jr., R.~C. 1998, \apj, 498, 541; {Koopmann}, R.~A. {et~al.} 1998, \apjl, 497, L75; ---. 2004, \apj, 613, 866; {Larson}, R.~B. {et~al.} 1980, ApJ, 237, 692; {Leroy}, A.~K. {et~al.} 2008, \aj, 136, 2782; {Lewis}, I. {et~al.} 2002, \mnras, 334, 673; {McGee}, S.~L. {et~al.} 2011, \mnras, 413, 996; {Papovich}, C. {et~al.} 2012, \apj, 750, 93; {Poggianti}, B.~M. {et~al.} 1999, ApJ, 518, 576; {Postman}, M. {et~al.} 2005, \apj, 623, 721; {Quilis}, V. {et~al.} 2000, Science, 288, 1617; {Saintonge}, A. {et~al.} 2008, \apjl, 685, L113; {Tran}, K.-V.~H. {et~al.} 2010, \apjl, 719, L126; {Weinmann}, S.~M. {et~al.} 2010, \mnras, 406, 2249}

%% %%%%%%%%%%%%%%%%%%%%%%%%%%%%%%%%%%%%%%%%%%%%%%%%%%%%%%%%%%%%%%%%%%%%%%%%%%%
%% \begin{multicols}{2}
%% %%\begin{thebibliography}{999}
%% \begin{thebibliography}{}
%% {\setlength{\itemsep}{-1mm}
%% %  \begin{minipage}[l]{4in}
%% \bibitem[{{Balogh} {et~al.}(2009){Balogh}, {McGee}, {Wilman}, {Bower}, {Hau},
%%   {Morris}, {Mulchaey}, {Oemler}, {Parker}, \& {Gwyn}}]{Balogh09}
%% {Balogh}, M.~L. {et~al.} 2009, \mnras, 398, 754

%% \bibitem[{{Balogh} {et~al.}(2000){Balogh}, {Navarro}, \& {Morris}}]{Balogh00}
%% ---. 2000, ApJ, 540, 113

%% \bibitem[{{Barger} {et~al.}(1996){Barger}, {Aragon-Salamanca}, {Ellis},
%%   {Couch}, {Smail}, \& {Sharples}}]{Barger96}
%% {Barger}, A.~J. {et~al.} 1996, MNRAS, 279, 1

%% \bibitem[{{Coil} {et~al.}(2008){Coil}, {Newman}, {Croton}, {Cooper}, {Davis},
%%   {Faber}, {Gerke}, {Koo}, {Padmanabhan}, {Wechsler}, \& {Weiner}}]{Coil08}
%% {Coil}, A.~L. {et~al.} 2008, \apj, 672, 153

%% \bibitem[{{Cox} {et~al.}(2008){Cox}, {Jonsson}, {Somerville}, {Primack}, \&
%%   {Dekel}}]{Cox08}
%% {Cox}, T.~J. {et~al.} 2008, \mnras, 384, 386

%% \bibitem[{{Dressler}(1980)}]{Dressler80b}
%% {Dressler}, A. 1980, ApJ, 236, 351

%% \bibitem[{{Ellingson} {et~al.}(2001){Ellingson}, {Lin}, {Yee}, \&
%%   {Carlberg}}]{Ellingson01}
%% {Ellingson}, E. {et~al.} 2001, \apj, 547, 609

%% \bibitem[{{Finn} {et~al.}(2010){Finn}, {Desai}, {Rudnick}, {Poggianti}, {Bell},
%%   {Hinz}, {Jablonka}, {Milvang-Jensen}, {Moustakas}, {Rines}, \&
%%   {Zaritsky}}]{Finn10}
%% {Finn}, R.~A. {et~al.} 2010, \apj, 720, 87

%% \bibitem[{{G{\' o}mez} {et~al.}(2003){G{\' o}mez}, {Nichol}, {Miller},
%%   {Balogh}, {Goto}, {Zabludoff}, {Romer}, {Bernardi}, {Sheth}, {Hopkins},
%%   {Castander}, {Connolly}, {Schneider}, {Brinkmann}, {Lamb}, {SubbaRao}, \&
%%   {York}}]{Gomez03}
%% {G{\' o}mez}, P.~L. {et~al.} 2003, \apj, 584, 210

%% \bibitem[{{Geach} {et~al.}(2006){Geach}, {Smail}, {Ellis}, {Moran}, {Smith},
%%   {Treu}, {Kneib}, {Edge}, \& {Kodama}}]{Geach06}
%% {Geach}, J.~E. {et~al.} 2006, \apj, 649, 661

%% \bibitem[{{Gunn} \& {Gott}(1972)}]{Gunn72}
%% {Gunn}, J.~E. {et~al.} 1972, ApJ, 176, 1

%% \bibitem[{{Hogg} {et~al.}(2004){Hogg}, {Blanton}, {Brinchmann}, {Eisenstein},
%%   {Schlegel}, {Gunn}, {McKay}, {Rix}, {Bahcall}, {Brinkmann}, \&
%%   {Meiksin}}]{Hogg04}
%% {Hogg}, D.~W. {et~al.} 2004, \apjl, 601, L29

%% \bibitem[{{Just} {et~al.}(2014){Just}, {Zaritsky}, {Cool}, {Moustakas},
%%   {Rudnick}, Clowe, Bian, {De Lucia}, {Aragon-Salamanca}, Desai, Finn,
%%   Halliday, Jablonka, Poggianti, \& White}]{Just14}
%% {Just}, D.~W. {et~al.} 2014, in preparation

%% \bibitem[{{Koyama} {et~al.}(2008){Koyama}, {Kodama}, {Shimasaku}, {Okamura},
%%   {Tanaka}, {Lee}, {Im}, {Matsuhara}, {Takagi}, {Wada}, \& {Oyabu}}]{Koyama08}
%% {Koyama}, Y. {et~al.} 2008, \mnras, 391, 1758

%% \bibitem[{{Le Floc'h} {et~al.}(2005){Le Floc'h}, {Papovich}, {Dole}, {Bell},
%%   {Lagache}, {Rieke}, {Egami}, {P{\'e}rez-Gonz{\'a}lez}, {Alonso-Herrero},
%%   {Rieke}, {Blaylock}, {Engelbracht}, {Gordon}, {Hines}, {Misselt}, {Morrison},
%%   \& {Mould}}]{LeFloch05}
%% {Le Floc'h}, E. {et~al.} 2005, \apj, 632, 169

%% \bibitem[{{Lewis} {et~al.}(2002){Lewis}, {Balogh}, {De Propris}, {Couch},
%%   {Bower}, {Offer}, {Bland-Hawthorn}, {Baldry}, {Baugh}, {Bridges}, {Cannon},
%%   {Cole}, {Colless}, {Collins}, {Cross}, {Dalton}, {Driver}, {Efstathiou},
%%   {Ellis}, {Frenk}, {Glazebrook}, {Hawkins}, {Jackson}, {Lahav}, {Lumsden},
%%   {Maddox}, {Madgwick}, {Norberg}, {Peacock}, {Percival}, {Peterson},
%%   {Sutherland}, \& {Taylor}}]{Lewis02}
%% {Lewis}, I. {et~al.} 2002, \mnras, 334, 673

%% \bibitem[{{McGee} {et~al.}(2011){McGee}, {Balogh}, {Wilman}, {Bower},
%%   {Mulchaey}, {Parker}, \& {Oemler}}]{McGee11}
%% {McGee}, S.~L. {et~al.} 2011, \mnras, 413, 996

%% \bibitem[{{Mihos} \& {Hernquist}(1994)}]{Mihos94}
%% {Mihos}, J.~C. {et~al.} 1994, \apjl, 425, L13

%% \bibitem[{{Patel} {et~al.}(2009){Patel}, {Kelson}, {Holden}, {Illingworth},
%%   {Franx}, {van der Wel}, \& {Ford}}]{Patel09}
%% {Patel}, S.~G. {et~al.} 2009, \apj, 694, 1349

%% \bibitem[{{Poggianti} {et~al.}(2009){Poggianti}, {Arag{\'o}n-Salamanca},
%%   {Zaritsky}, {De Lucia}, {Milvang-Jensen}, {Desai}, {Jablonka}, {Halliday},
%%   {Rudnick}, {Varela}, {Bamford}, {Best}, {Clowe}, {Noll}, {Saglia},
%%   {Pell{\'o}}, {Simard}, {von der Linden}, \& {White}}]{Poggianti09}
%% {Poggianti}, B.~M. {et~al.} 2009, \apj, 693, 112

%% \bibitem[{{Poggianti} {et~al.}(2006){Poggianti}, {von der Linden}, {De Lucia},
%%   {Desai}, {Simard}, {Halliday}, {Arag{\'o}n-Salamanca}, {Bower}, {Varela},
%%   {Best}, {Clowe}, {Dalcanton}, {Jablonka}, {Milvang-Jensen}, {Pello},
%%   {Rudnick}, {Saglia}, {White}, \& {Zaritsky}}]{Poggianti06}
%% ---. 2006, \apj, 642, 188

%% \bibitem[{{Rudnick} {et~al.}(2009){Rudnick}, {von der Linden}, {Pell{\'o}},
%%   {Arag{\'o}n-Salamanca}, {Marchesini}, {Clowe}, {DeLucia}, {Halliday},
%%   {Jablonka}, {Milvang-Jensen}, {Poggianti}, {Saglia}, {Simard}, {White}, \&
%%   {Zaritsky}}]{Rudnick09}
%% {Rudnick}, G. {et~al.} 2009, \apj, 700, 1559

%% \bibitem[{{Springel} {et~al.}(2005){Springel}, {Di Matteo}, \&
%%   {Hernquist}}]{Springel05}
%% {Springel}, V. {et~al.} 2005, \apjl, 620, L79

%% \bibitem[{{Strateva} {et~al.}(2001){Strateva}, {Ivezi{\' c}}, {Knapp},
%%   {Narayanan}, {Strauss}, {Gunn}, {Lupton}, {Schlegel}, {Bahcall}, {Brinkmann},
%%   {Brunner}, {Budav{\' a}ri}, {Csabai}, {Castander}, {Doi}, {Fukugita}, {Gy{\H
%%   o}ry}, {Hamabe}, {Hennessy}, {Ichikawa}, {Kunszt}, {Lamb}, {McKay},
%%   {Okamura}, {Racusin}, {Sekiguchi}, {Schneider}, {Shimasaku}, \&
%%   {York}}]{Strateva01}
%% {Strateva}, I. {et~al.} 2001, \aj, 122, 1861

%% \bibitem[{{van Dokkum} {et~al.}(1998){van Dokkum}, {Franx}, {Kelson},
%%   {Illingworth}, {Fisher}, \& {Fabricant}}]{vandokkum98}
%% {van Dokkum}, P.~G. {et~al.} 1998, \apj, 500, 714

%% \bibitem[{{White} {et~al.}(2005){White}, {Clowe}, {Simard}, {Rudnick}, {de
%%   Lucia}, {Arag{\'o}n-Salamanca}, {Bender}, {Best}, {Bremer}, {Charlot},
%%   {Dalcanton}, {Dantel}, {Desai}, {Fort}, {Halliday}, {Jablonka}, {Kauffmann},
%%   {Mellier}, {Milvang-Jensen}, {Pell{\'o}}, {Poggianti}, {Poirier},
%%   {Rottgering}, {Saglia}, {Schneider}, \& {Zaritsky}}]{White05}
%% {White}, S.~D.~M. {et~al.} 2005, \aap, 444, 365
%% }
%% \end{thebibliography}{}
%% \end{multicols}

%\clearpage
%\cfoot[]{D-\thepage}
%\renewcommand{\thepage}{E--\arabic{page}}

%\setcounter{page}{1}
\clearpage
\bibliographystyle{/Users/grudnick/LaTeX/Bibtex/apj1lim}
\bibliography{references}




\end{document}
